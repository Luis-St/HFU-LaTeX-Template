% # HFU LaTeX Vorlage
%
% Autor: Luis Staudt
% Korrigiert für HFU-Compliance
% Lizenz: Free to use, but please keep this header
% Hinweis: Wir übernehmen keine Garantie für die Richtigkeit der Vorlage


% ##################################################
% ENCODING
% ##################################################
\usepackage{cmap}               % PDF character encoding
\usepackage[T1]{fontenc}        % 8-bit font encoding
\usepackage[utf8]{inputenc}     % UTF-8 input encoding
\usepackage[german]{babel}      % Sprache festlegen


% ##################################################
% DOCUMENT VARIABLES
% ##################################################
\newcommand{\thesistype}{Projektarbeit/Praxissemesterbericht/Bachelorarbeit/Masterarbeit}
\newcommand{\studyProgram}{Informatik}
\newcommand{\thesistitle}{Title of the Thesis}
\newcommand{\thesissubtitle}{Subtitle of the Thesis}
\newcommand{\supervisorname}{Prof. Dr. Supervisor Name}
\newcommand{\cosupervisorname}{Prof. Dr. Co-Supervisor Name}
\newcommand{\deadline}{\today}

\newcommand{\authorname}{Your Name}
\newcommand{\matriculationNumber}{xxxxxx}
\newcommand{\streetName}{Street Name, House Number}
\newcommand{\postalCode}{12345}
\newcommand{\city}{City Name}
\newcommand{\mail}{user-name@stud.hs-furtwangen.de}


% ##################################################
% GENERAL
% ##################################################
\usepackage{scrhack} % better KOMA adaptions
\usepackage[table]{xcolor} % color support
\usepackage{chngcntr} % for renumbering stuff
\usepackage{amsthm}
\usepackage{lipsum} % lorem ipsum generator


% ##################################################
% PDF SETTINGS
% ##################################################
\usepackage[
	colorlinks=false,
	linkcolor=black,
	citecolor=black,
	filecolor=black,
	urlcolor=black,
	bookmarks=true,
	bookmarksopen=true,
	bookmarksopenlevel=3,
	bookmarksnumbered,
	plainpages=false,
	pdfpagelabels=true,
	hyperfootnotes,
	pdftitle ={\thesistitle},
	pdfauthor={\authorname},
	pdfcreator={\authorname}
]{hyperref}


% ##################################################
% FONTS AND SPACING
% ##################################################
\usepackage{lmodern}                        % Latin Modern Schrift
\renewcommand{\familydefault}{\rmdefault}   % Serifenschrift
\usepackage[onehalfspacing]{setspace}       % 1.5 Zeilenabstand
\raggedbottom   % don't stretch spacing to fit page length

\usepackage{anyfontsize} % use any font size

\addtokomafont{chapter}{\fontsize{14}{18}\selectfont\bfseries}    % 14pt fett für Hauptkapitel
\addtokomafont{section}{\fontsize{12}{15}\selectfont\bfseries}    % 12pt fett für Unterkapitel (Ebene 2)
\addtokomafont{subsection}{\fontsize{12}{15}\selectfont}          % 12pt normal für Unterkapitel (Ebene 3)
\addtokomafont{caption}{\fontsize{12}{15}\selectfont}             % 12pt für Bildunterschriften

\addtokomafont{pagehead}{\fontsize{12}{15}\selectfont\bfseries}

% url font style
\usepackage{relsize}
\renewcommand*{\UrlFont}{\ttfamily\smaller\relax}


% ##################################################
% PAGE FORMATTING
% ##################################################
% Page layout / Seitenränder - HFU-konform
\usepackage[
	bindingoffset=1.5cm,    % Bundsteg
	inner=2.5cm,            % Innenrand
	outer=2.5cm,            % Außenrand
	top=3cm,                % Oberrand
	bottom=2cm              % Unterrand
]{geometry}

% KORREKTUR: Befehl für symmetrische Ränder (für Titelblatt)
\newcommand{\symmetricmargins}{%
	\newgeometry{
		left=2.5cm,
		right=2.5cm,
		top=3cm,
		bottom=2cm
	}%
}

% Befehl zum Zurücksetzen auf HFU-Ränder
\newcommand{\hfumargins}{%
	\newgeometry{
		bindingoffset=1.5cm,
		inner=2.5cm,
		outer=2.5cm,
		top=3cm,
		bottom=2cm
	}%
}


% Command for creating new pages
\newcommand{\newchapterpage}{\cleardoublepage}
\newcommand{\newsectionpage}{\clearpage}
\newcommand{\blankpage}{\newpage\thispagestyle{empty}\mbox{}\newpage}

% Command for emphasis
\newcommand{\emphtext}[1]{\textbf{#1}}
\newcommand{\codetext}[1]{\texttt{#1}}

% Custom commands for common phrases
\newcommand{\chapterref}[1]{Kapitel~\ref{#1}}
\newcommand{\sectionref}[1]{Abschnitt~\ref{#1}}

\newcommand{\figureref}[1]{Bild~\ref{#1}}
\newcommand{\tableref}[1]{Tabelle~\ref{#1}}
\newcommand{\listingref}[1]{Code-Block~\ref{#1}}
\newcommand{\equationref}[1]{Gleichung~\ref{#1}}


% Page header
\usepackage[
	headsepline,        % seperator line beneath page header on normal pages
	plainheadsepline    % seperator line beneath page header on pages like ToC
]{scrlayer-scrpage}
\clearpairofpagestyles                  % clear default settings
\ohead*{\thepage}                       % page number außen
\ihead*{\leftmark}                      % chapter name innen


% ##################################################
% IMAGES AND FIGURES
% ##################################################
\usepackage{graphicx}       % support for including images
\graphicspath{{images/}}    % default path

% simple numbering without chapter
\renewcommand{\thefigure}{\arabic{figure}}
\counterwithout{figure}{chapter}

\usepackage{wrapfig}       % wrap text around figures


% ##################################################
% TABLES
% ##################################################
\usepackage{booktabs}   % beautiful table style
\usepackage{multirow}   % multi row and multi column table functionality

% simple numbering without chapter
\renewcommand{\thetable}{\arabic{table}}
\counterwithout{table}{chapter}


% ##################################################
% SOURCE CODE LISTINGS
% ##################################################
\usepackage{listings}
\usepackage{beramono}   % use a typewriter font which supports bold characters

\renewcommand{\lstlistlistingname}{Verzeichnis der Code-Blöcke}
\renewcommand{\lstlistingname}{Code-Block}
\newcommand{\listoflolentryname}{\lstlistingname}

% define colors for source code highlighting
\definecolor{codegreen}{rgb}{0,0.6,0}
\definecolor{codegray}{rgb}{0.5,0.5,0.5}
\definecolor{codepurple}{rgb}{0.5,0,0.33}
\definecolor{codepurblue}{rgb}{0.16,0.0,1.0}
\definecolor{backcolour}{rgb}{0.95,0.95,0.92}

% set source code style
\lstdefinestyle{codestyle}{
	backgroundcolor=\color{backcolour},
	commentstyle=\color{codegreen},
	keywordstyle=\bfseries\color{codepurple},
	numberstyle=\tiny\color{codegray},
	stringstyle=\color{codepurblue},
	basicstyle=\fontsize{10}{12}\ttfamily,  % 10pt für Code (HFU Tabellengröße)
	breakatwhitespace=false,
	breaklines=true,
	captionpos=b,
	keepspaces=true,
	numbers=left,
	numbersep=5pt,
	showspaces=false,
	showstringspaces=false,
	showtabs=false,
	tabsize=2,
	escapeinside={(*@}{@*)}
}
\lstset{style=codestyle, numberbychapter=false}


% ##################################################
% TABLE OF CONTENTS
% ##################################################
\KOMAoptions{toc=chapterentrydotfill}       % dotted lines for chapters
\addtokomafont{chapterentry}{\normalfont}   % use normal font for zentrierung entries
\setuptoc{toc}{totoc}                       % add ToC to ToC

% spacing
\DeclareTOCStyleEntry[beforeskip=0cm]{chapter}{chapter}
\DeclareTOCStyleEntry[beforeskip=0cm]{section}{section}
\DeclareTOCStyleEntry[beforeskip=0cm]{default}{subsection}

% colons after entry names
\BeforeStartingTOC[lof]{\def\autodot{:}}
\BeforeStartingTOC[lot]{\def\autodot{:}}
\BeforeStartingTOC[lol]{\def\autodot{:}}


% ##################################################
% BIBLIOGRAPHY
% ##################################################
\usepackage{csquotes} % context sensitive quotation
\usepackage[backend=biber,style=ieee,sorting=none]{biblatex} % bibliography support
\addbibresource{bibtex.bib} % bibliography file

% HFU-konform: 1,5-zeiliger Abstand zwischen Einträgen, 1-zeilig innerhalb der Einträge
\setlength\bibitemsep{1.5\baselineskip} % 1,5-zeiliger Abstand zwischen Einträgen
\renewcommand*{\bibsetup}{\singlespacing} % 1-zeilig innerhalb der Einträge

\setcounter{biburlnumpenalty}{9000} % break URLs on numbers
\setcounter{biburllcpenalty}{9000}  % break URLs on lower case letters
\setcounter{biburlucpenalty}{9000}  % break URLs on upper case letters


% ##################################################
% ABBREVIATIONS
% ##################################################
\usepackage[printonlyused]{acronym}


% ##################################################
% APPENDIX
% ##################################################
\usepackage[title,titletoc]{appendix}

% appendix chapter
\newcommand{\appendixchapter}[1]{
	\newchapterpage
	\pagenumbering{arabic}
	\renewcommand{\thepage}{\thechapter-\arabic{page}}
	\chapter{#1}
}

% insert monthly report pdf as picture in order to keep page header
\newcommand{\monthlyreport}[2]{
	\section{#1}
	\centering
	\includegraphics[trim=55 35 55 35,clip,width=1\textwidth]{#2}
	\newsectionpage
}


% ##################################################
% Theoreme
% ##################################################
\newtheorem{beispiel}{Beispiel}

% Umgebung fuer These
\newtheorem{these}{These}

% Umgebung fuer Definitionen
\newtheorem{definition}{Definition}


% ##################################################
% MISC
% ##################################################
% better referencing of images, tables, etc.
\usepackage[nameinlink, noabbrev]{cleveref}
