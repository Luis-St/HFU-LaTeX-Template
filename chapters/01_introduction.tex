% Sample Chapter: Introduction
\chapter{Einleitung}
\label{ch:introduction}

Diese Einleitung führt in das Thema der Arbeit ein und stellt die Problemstellung vor.
Folgende Punkte sollten hier behandelt werden:

\section{Abkürzungen und Begriffe}
\label{sec:abbreviations_and_terms}
In diesem Abschnitt werden wichtige Abkürzungen und Begriffe definiert, die in der Arbeit verwendet werden.
\acl{JVM} steht für Java Virtual Machine, eine Laufzeitumgebung für Java-Anwendungen.
\acs{HTML} bezeichnet Hypertext Markup Language, die Standard-Auszeichnungssprache für Webseiten.

\section{Motivation}
\label{sec:motivation}

Die Motivation erklärt, warum das behandelte Thema relevant ist und welche praktischen oder theoretischen Probleme gelöst werden sollen.

% Example citation
Wie bereits in der Literatur~\cite{example_book} gezeigt wurde, ist dieses Problem von großer Bedeutung.
Weitere Studien~\cite{example_article} bestätigen diese Einschätzung und zeigen die praktische Relevanz auf.

\section{Problemstellung}
\label{sec:problem_statement}

Die Problemstellung beschreibt konkret, welches Problem in dieser Arbeit behandelt wird.
Sie sollte präzise und mit klaren Abgrenzungen formuliert werden.

Aktuelle Forschungsarbeiten~\cite{example_conference} haben gezeigt, dass bestehende Ansätze verschiedene Limitierungen aufweisen.
Insbesondere die von \textcite{example_article} identifizierten Herausforderungen erfordern neue Lösungsansätze.

\section{Zielsetzung}
\label{sec:objectives}

Die Zielsetzung definiert, was mit dieser Arbeit erreicht werden soll.
Die Ziele sollten:

\begin{itemize}
	\item Messbar sein
	\item Realistisch erreichbar sein
	\item Einen klaren Beitrag leisten
\end{itemize}

Basierend auf den Erkenntnissen aus~\cite{example_report} werden in dieser Arbeit folgende Ziele verfolgt:

\begin{enumerate}
	\item Entwicklung eines verbesserten Ansatzes
	\item Evaluierung der Leistungsfähigkeit
	\item Vergleich mit bestehenden State-of-the-Art-Methoden
\end{enumerate}

\section{Aufbau der Arbeit}
\label{sec:structure}

Dieser Abschnitt gibt einen Überblick über den Aufbau der Arbeit:

\begin{description}
	\item[\chapterref{ch:fundamentals}] behandelt die theoretischen Grundlagen
	\item[Weitere Kapitel] folgen der logischen Struktur der Problemlösung
\end{description}

\begin{figure}[htbp]
	\centering
	\includegraphics[width=0.8\textwidth]{hfu}
	\caption{Beispielabbildung der Gesamtsystemarchitektur}
	\label{fig:example}
\end{figure}

Wie in \figureref{fig:example} dargestellt, besteht das System aus mehreren Komponenten, die miteinander interagieren.

% Example table
\begin{table}[htbp]
	\centering
	\caption{Vergleich verschiedener Ansätze}
	\label{tab:comparison}
	\begin{tabular}{@{}lcc@{}}
		\toprule
		\emphtext{Kriterium} & \emphtext{Ansatz 1} & \emphtext{Ansatz 2} \\
		\midrule
		Genauigkeit & 95\% & 87\% \\
		Geschwindigkeit & 1,2s & 2,1s \\
		Speicherverbrauch & 512MB & 256MB \\
		\bottomrule
	\end{tabular}
\end{table}

\tableref{tab:comparison} zeigt, dass Ansatz 1 eine bessere Genauigkeit erreicht, aber mehr Speicher benötigt.

% Example code listing
\begin{lstlisting}[language=Python, caption=Beispiel Python-Code, label=lst:example]
def berechne_genauigkeit(vorhersagen, grundwahrheit):
    """Berechnet die Genauigkeit von Vorhersagen."""
    korrekt = sum(v == gw for v, gw in zip(vorhersagen, grundwahrheit))
    gesamt = len(vorhersagen)
    return korrekt / gesamt if gesamt > 0 else 0.0

# Verwendungsbeispiel
pred = [1, 0, 1, 1, 0]
wahrheit = [1, 0, 0, 1, 0]
genauigkeit = berechne_genauigkeit(pred, wahrheit)
print(f"Genauigkeit: {genauigkeit:.2%}")
\end{lstlisting}

Der Code in \listingref{lst:example} demonstriert, wie Genauigkeitsmetriken berechnet werden können.

\section{Forschungsfragen}
\label{sec:research_questions}

Diese Arbeit behandelt folgende Forschungsfragen:

\begin{enumerate}
	\item Wie kann \emphtext{Methode X} verbessert werden, um eine bessere Leistung zu erzielen?
	\item Was sind die Limitierungen aktueller Ansätze in \emphtext{Bereich Y}?
	\item Wie schneidet die vorgeschlagene Lösung im Vergleich zu bestehenden State-of-the-Art-Methoden ab?
\end{enumerate}

Diese Fragen werden im Verlauf der Arbeit systematisch beantwortet, mit besonderem Fokus auf \emphtext{Forschungsfrage 1}.

\section{Methodik}
\label{sec:methodology}

Die methodische Herangehensweise orientiert sich an bewährten Praktiken des Software Engineering~\cite{example_web} und berücksichtigt aktuelle Standards~\cite{example_standard}.
Ähnliche Ansätze wurden bereits in~\cite{example_thesis} erfolgreich angewendet.
