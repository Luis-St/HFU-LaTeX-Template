% Template for additional chapters
\chapter{Grundlagen}
\label{ch:fundamentals}

Dieses Kapitel liefert die theoretischen Grundlagen, die für das Verständnis der in dieser Arbeit vorgestellten Arbeiten notwendig sind.

\section{Grundbegriffe}
\label{sec:basic_concepts}

Hier werden die fundamentalen Konzepte eingeführt, die für diese Arbeit wesentlich sind.
Die Terminologie orientiert sich an etablierten Standards~\cite{example_standard} und aktueller Fachliteratur~\cite{example_book}.

\subsection{Definition wichtiger Begriffe}
\label{subsec:definitions}

\begin{description}
	\item[Softwareentwicklung] Systematischer Prozess zur Erstellung von Softwareprodukten unter Berücksichtigung von Qualitätsaspekten~\cite{example_book}.
	\item[Maschinelles Lernen] Teilbereich der künstlichen Intelligenz, der sich mit der automatischen Erkennung von Mustern in Daten beschäftigt~\cite{example_conference}.
	\item[Cloud Computing] Bereitstellung von IT-Ressourcen über das Internet auf Basis flexibler Nutzungsmodelle~\cite{example_report}.
\end{description}

\section{Theoretischer Hintergrund}
\label{sec:theoretical_background}

Dieser Abschnitt behandelt den theoretischen Hintergrund des Forschungsbereichs und stützt sich auf aktuelle wissenschaftliche Erkenntnisse~\cite{example_article}.

\subsection{Mathematische Grundlagen}
\label{subsec:math_foundations}

Die mathematischen Grundlagen dieser Arbeit basieren auf etablierten Konzepten der Informatik. Zum Beispiel:

\begin{equation}
	f(x) = ax^2 + bx + c
	\label{eq:quadratic}
\end{equation}

\equationref{eq:quadratic} zeigt eine quadratische Funktion, wobei $a$, $b$ und $c$ Konstanten sind.
Diese Form der mathematischen Modellierung ist in der Literatur~\cite{example_book} ausführlich beschrieben.

Für komplexere Berechnungen gilt:

\begin{equation}
	\sum_{i=1}^{n} x_i = \frac{n(n+1)}{2}
	\label{eq:sum}
\end{equation}

\subsection{Algorithmische Grundlagen}
\label{subsec:algorithmic_foundations}

Die algorithmischen Grundlagen orientieren sich an bewährten Verfahren der Informatik~\cite{example_conference}:

\begin{lstlisting}[language=Python, caption=Beispielalgorithmus für binäre Suche, label=lst:algorithm]
def binaere_suche(arr, ziel):
    """Fuehrt eine binaere Suche in einem sortierten Array durch.
    
    Args:
        arr: Sortiertes Array von Elementen
        ziel: Zu suchendes Element
        
    Returns:
        Index des Elements oder -1 falls nicht gefunden
    """
    links, rechts = 0, len(arr) - 1

    while links <= rechts:
        mitte = (links + rechts) // 2
        if arr[mitte] == ziel:
            return mitte
        elif arr[mitte] < ziel:
            links = mitte + 1
        else:
            rechts = mitte - 1

    return -1  # Nicht gefunden
\end{lstlisting}

Der in \listingref{lst:algorithm} dargestellte Algorithmus hat eine Zeitkomplexität von $O(\log n)$ und ist damit deutlich effizienter als eine lineare Suche.

\section{Technologie-Stack}
\label{sec:technology_stack}

Die Auswahl der verwendeten Technologien basiert auf einer systematischen Bewertung verfügbarer Alternativen.
Die Bewertungskriterien orientieren sich an den in~\cite{example_report} beschriebenen Best Practices.

\begin{table}[htbp]
	\centering
	\caption{Technologievergleich basierend auf definierten Kriterien}
	\label{tab:tech_comparison}
	\begin{tabular}{@{}lccc@{}}
		\toprule
		\emphtext{Technologie} & \emphtext{Leistung} & \emphtext{Benutzerfreundlichkeit} & \emphtext{Community} \\
		\midrule
		Python & Hoch & Hoch & Groß \\
		Java & Sehr hoch & Mittel & Groß \\
		JavaScript & Mittel & Hoch & Sehr groß \\
		C++ & Sehr hoch & Niedrig & Mittel \\
		\bottomrule
	\end{tabular}
\end{table}

\tableref{tab:tech_comparison} zeigt, dass Python die beste Balance zwischen Leistung und Benutzerfreundlichkeit bietet, während Java die höchste Leistung aufweist.

\section{Stand der Technik}
\label{sec:state_of_art}

Der aktuelle Stand der Technik wird durch verschiedene Ansätze und Implementierungen geprägt.
Besonders hervorzuheben sind die Arbeiten von~\textcite{example_conference}, die einen innovativen Ansatz für maschinelles Lernen in der Softwareentwicklung vorschlagen.

Online-Ressourcen~\cite{example_web} bieten zusätzliche Einblicke in praktische Implementierungsaspekte und aktuelle Entwicklungen.

\subsection{Vergleich bestehender Ansätze}
\label{subsec:existing_approaches}

Die Literaturanalyse zeigt verschiedene Kategorien von Lösungsansätzen:

\begin{enumerate}
	\item \emphtext{Traditionelle Ansätze:} Basieren auf etablierten Algorithmen~\cite{example_book}
	\item \emphtext{Machine Learning-basierte Ansätze:} Nutzen moderne KI-Techniken~\cite{example_conference}
	\item \emphtext{Hybrid-Ansätze:} Kombinieren verschiedene Methodiken~\cite{example_article}
\end{enumerate}

\section{Identifizierte Forschungslücken}
\label{sec:research_gaps}

Basierend auf der Analyse der bestehenden Literatur lassen sich folgende Forschungslücken identifizieren:

\begin{itemize}
	\item \emphtext{Skalierbarkeit:} Bestehende Ansätze~\cite{example_report} zeigen Limitierungen bei großen Datensätzen
	\item \emphtext{Generalisierbarkeit:} Viele Lösungen~\cite{example_thesis} sind domänenspezifisch und schwer übertragbar
	\item \emphtext{Effizienz:} Trade-offs zwischen Genauigkeit und Geschwindigkeit sind noch nicht optimal gelöst~\cite{example_article}
\end{itemize}

\section{Zusammenfassung}
\label{sec:fundamentals_summary}

Dieses Kapitel hat die theoretischen Grundlagen für die in dieser Arbeit vorgestellte Forschung etabliert.
Die hier eingeführten Schlüsselkonzepte bilden die Basis für die nachfolgenden Kapitel und werden insbesondere bei der Entwicklung der eigenen Lösung referenziert.

Die identifizierten Forschungslücken motivieren den in \chapterref{ch:introduction} vorgestellten Ansatz und zeigen die Relevanz der geplanten Forschungsarbeit auf.
